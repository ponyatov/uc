% ebook header
\documentclass[oneside,10pt]{article}
\usepackage[paperwidth=118.8mm,paperheight=68.2mm,margin=2mm]{geometry}
\renewcommand{\familydefault}{\sfdefault}\normalfont
\usepackage[unicode,colorlinks=true]{hyperref}
\usepackage[pdftex]{graphicx}
\newcommand{\fig}[3]{\noindent\includegraphics[#3]{#2}\textbf{\ #1}}
%% Cyrillization
\usepackage[utf8]{inputenc}
\usepackage[english,russian]{babel}
%% misc
\newcommand{\note}[1]{\,\footnote{\ #1}}
\newcommand{\email}[1]{$<$\href{mailto:#1}{#1}$>$}
%% lang
\newcommand{\cpp}{$C^+_+$}
%% prog
\newcommand{\prog}[1]{\tetxbf{#1}}
\newcommand{\file}[1]{\textit{#1}}
\newcommand{\make}{\prog{make}}
\newcommand{\git}{\prog{git}}
\newcommand{\gvim}{\prog{(g)Vim}}
\newcommand{\flex}{\prog{flex}}
\newcommand{\lex}{\prog{lex}}
\newcommand{\yacc}{\prog{yacc}}
\newcommand{\bison}{\prog{bison}}
\newcommand{\gpp}{\prog{g++}}
%% listings
\usepackage{listings}
\lstset{
basicstyle=\small,
frame=single,
tabsize=4,
}
\newcommand{\lst}[2]{\lstinputlisting[title=#1]{#2}}

\title{uCompiler: микрокомпилятор}
\author{\copyright\ Dmitry Ponyatov \email{dponyatov@gmail.com}}

\begin{document}
\maketitle
\begin{abstracts}
Рассмотрена реализация компилятора для простейшего арифметического языка на
\flex/\bison/\cpp.
\end{abstracts}
\clearpage\tableofcontents\clearpage

\section{Необходимое программное обеспечение}

\begin{itemize}
  \item пакет {mingw}\note{\url{http://www.mingw.org/}}: нужны пакеты
  \gpp\ \make\ \flex\ \bison
  \item рекомендуется текстовый редактор
  \gvim\note{\url{http://vim.sourceforge.net/download.php#pc}}
\end{itemize}

\section{Структура проекта (lexical skeleton)}

\begin{tabular}{l l l}
src.src & script & тестовый скрипт на нашем языке \\
log.log & & лог выполнения компилятора \\
ypp.ypp & \yacc & описание синтаксиса (грамматика) \\
lpp.lpp & \lex & регулярные выражения для лексем \\
hpp.hpp & \cpp & общие хедеры \\
cpp.cpp & \cpp & \\
.gitignore & \git & маски файлов не включаемых в проект \\
bat.bat & \gvim & helper для запуска редактора \\
Makefile & \make & скрипт сборки проекта \\
\end{tabular}

\lst{\file{core.src}: ядро компилятора}{doc/core.src}
\lst{\file{files.src}: файлы проекта}{doc/files.src}

\subsection{Структура компилятора}

\fig{}{tmp/compiler.pdf}{height=0.9\textheight}

\subsection{Сборка проекта утилитой [mingw32-]\make}

\lst{Makefile}{doc/00.mk}

\end{document}
